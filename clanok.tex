% Metódy inžinierskej práce

\documentclass[10pt,twoside,slovak,a4paper]{article}

\usepackage[slovak]{babel}
%\usepackage[T1]{fontenc}
\usepackage[IL2]{fontenc} % lepšia sadzba písmena Ľ než v T1
\usepackage[utf8]{inputenc}
\usepackage{graphicx}
\usepackage{url} % príkaz \url na formátovanie URL
\usepackage{hyperref} % odkazy v texte budú aktívne (pri niektorých triedach dokumentov spôsobuje posun textu)

\usepackage{cite}
\usepackage{pdfpages}
%\usepackage{times}

\pagestyle{headings}

\title{Zlepšenie matematických schopností študenta pomocou logických počítačových hier\thanks{Semestrálny projekt v predmete Metódy inžinierskej práce, ak. rok 2022/23, vedenie: Ing. Vladimír Mlynarovič, PhD.}} % meno a priezvisko vyučujúceho na cvičeniach

\author{Filip Chromek\\[2pt]
	{\small Slovenská technická univerzita v Bratislave}\\
	{\small Fakulta informatiky a informačných technológií}\\
	{\small \texttt{xchromek@stuba.sk}}
	}

\date{\small 11. október 2022} % upravte



\begin{document}

\maketitle

\begin{abstract}

Iba málo študentov v dnešnej dobe má rado matematiku. Na druhej strane má veľa mladých ľudí rado hranie hier. Či už hier na počítači alebo na mobilnom telefóne.

Počítačové hry zväčša využívajú prirodzenú zvedavosť a súťaživosť ľudskej mysle, aby presvedčili človeka, aby sa im venoval. Hry taktiež využívajú zvýšené vylučovanie endorfínov a dopamínov počas hrania, čo vedie k ešte väčšiemu záujmu hrať.

Tento fakt je vhodné využiť pre niečo dobré. Pri hraní hier, ktorým obsahom sú informácie z reálneho sveta, si daný hráč dokáže podvedome tieto informácie zapamätať. V hrách musí tieto informácie využívať pre dosiahnutie progresu, musí sa nad nimi zamýšľať a vďaka tomu sa mu „uložia“ v pamäti, aj keď sa ich pôvodne učiť nechcel. Takýto štýl učenia je veľmi účinný najmä pri vedomostiach z oblasti matematiky, keďže sa ich väčšinou nedá „naučiť naspamäť“, ale je potrebné sa nad nimi zamyslieť.
\end{abstract}



\section{Úvod}


Využitie počítačových hier vo vzdelávaní má veľký potenciál vďaka veľkému záujmu dnešnej generácie o ich používanie vo voľnom čase. Bohužial zatiaľ nevidíme veľký záujem skôl a vzdelávacích inšitúcií o ich aplikovanie vo výučbe. Jeden z problémov, pre ktorý zatiaľ hry nie sú aktívne využívané vo vzdelávaní, je napríklad ich podceňovanie. Hlavným cieľom vzdelávania pomocou počítačových hier je učenie hrou, dobrá nálada a prirodzená motivácia. Je dokázané, že tieto faktory pomáhajú pri učení \cite{stranka1}. Jedným z ďalších problémov je nedostatok hier vhodných pre výučbu. V dnešnej dobe už prebieha niekoľko globálnych projektov\cite{stranka1} zameraných na vývoj edukačných počítačových hier, zameraných najmä na matematiku. Ďalšou výhodou použitia počítačových hier vo vzdelávaní je to, že vďaka endorfínom vylúčeným pri hraní si študent vytvorí pozitívny prístup k vzdelávaniu. 
V tomto článku budeme porovnávať výhody \ref{vyhodynevyhodyhier:vyhody} a nevýhody \ref{vyhodynevyhodyhier:vyhody} vzdelávania pomocou hier verzus výhody \ref{vyhodynevyhodytrad:vyhody} a nevýhody \ref{vyhodynevyhodytrad:nevyhody} vzdelávania tradičného typu. Taktiež sa budeme zaoberať najznámejšími hrami \ref{edukacnehry}, ktoré sa dajú využiť vo vzdelávaní.


\section{Štúdie zaoberajúce sa problematikou vzdelávania pomocou počítačových hier}\label{studie}
Touto problematikou sa k dnešnému dňu zaoberalo viacero štúdií. Tento článok čerpá hlavne zo štúdií Tobias and Fletcher, 2011a; Tobias, Fletcher, Dai, and Wind, 2011,\cite{strankazdroj1} ale taktiež zo štúdie z roku 2022 z Univerzity v Kolíne\cite{strankazdroj2}. Všetky tieto štúdie vidia veľký potenciál vo vzdelávaní pomocou počítačových hier a odporúčajú ich používanie v rámci vyučovania. V štúdiach sa stretávame s výrazmi ako podvedomé učenie sa, efektívne sprostredkovanie informácií, vizuálne učenie sa a potreba zamyslieť sa nad problémom. V štúdiach sa taktiež spomína potreba správneho rozvrhnutia času pre hry, správne nadizajnované edukačné hry a taktiež to, že by hry nemali byť jediným spôsobom vzdelávania.


\section{Výhody a nevýhody použitia počítačových hier vo výučbe} \label{vyhodynevyhodyhier}

Hranie počítačových hier je pre hráča v drvivej väčšine času zábavné. Je to spôsobené týmito faktormi:
\begin{itemize}
\item Hry majú príbeh
\item Hry majú cieľ
\item Hry sú interaktívne
\item Hráč musí riešiť hádanky, aby sa posunul v deji
\item Hráč je je odmenený dávkou endorfínou, keď vyhrá
\end{itemize}


\subsection{Výhody}\label{vyhodynevyhodyhier:vyhody} 
Medzi výhody počítačových hier patria napríklad tieto:
\begin{itemize}
\item Rozvíjajú kreatívne myslenie
\item Učenie hrou a zábava pri učení
\item Pri hrách je potrebné využiť viacero zmyslov (zrak, sluch, hmat), čo pomáha so zapamätaním si faktu
\item Potrebné spájanie si súvislostí a logické myslenie
\item Multiplayerové hry zlepšujú spoluprácu s inými ľuďmi
\end{itemize}




\subsection{Nevýhody}\label{vyhodynevyhodyhier:nevyhody} 
\begin{itemize}
\item Zdravotné problémy spôsobené nedostatkom pohybu
\item Možnosť rozvoja závislosti
\item Študent nie je motivovaný čítať knihy
\item Nedostatok sociálneho kontaktu
\item Na internete sa nachádzajú aj nevhodné materiály a hry
\end{itemize}



\section{Výhody a nevýhody vzdelávania pomocou tradičných metod} \label{vyhodynevyhodytrad}

\subsection{Výhody}\label{vyhodynevyhodytrad:vyhody} 
\begin{itemize}
\item Relatívne dobrá organizácia času
\item Orientácia na informácie - stanovené ciele
\item Sociálny kontakt
\end{itemize}



\subsection{Nevýhody}\label{vyhodynevyhodytrad:nevyhody} 
\begin{itemize}
\item Obmedzenie prestorom triedy (školy) - nemožnosť pohybu
\item Veľký dôraz sa kladie na pamäťové učenie, deti nie sú nútené rozmýšlať
\item Autoritatívne postavenie učiteľa
\end{itemize}

\section{Edukačné hry\cite{strankahry}}\label{edukacnehry}
\subsection{Hry zamerané na matematiku}\label{edukacnehry:hrymat} 
\subsubsection{Brainiversity}
Brainiversity je hra zameraná na viaceré schopnosti ako pamäť, matematické schopnosti, jazykové schopnosti a analýzu problému.
V tejto hre je viacero aktivít. V jednej, s názvom Add It Up musí hráč vyriešiť čo najviac matematických rovníc v určenom čase.
Hra zaznamenáva skóre jednotlivých aktivít a vďaka tomu je možné pozorovať zlepšenie alebo zhoršenie výsledku.
\subsubsection{Math Blaster Series}
Math Blaster Series je séria počítačových hier poskytuje hodnotné nástroje pre výučbu sčítania, odčítania, násobenia a delenia. Hráčov počas hrania sprevádza pes s menom Spot.
Táto hra je vhodná hlavne pre mladších žiakov.
\subsubsection{Universe Sandbox}
Universe Sandbox pomáha okrem matematických schopností rozvíjať aj vedomosti o fyzike.
Hráč má možnosť vytvoriť si vlastnú sústavu planét, vďaka čomu získava vedomosti.
\subsection{Hry zamerané na iné predmety}\label{edukacnehry:hryine}
\subsubsection{Minecraft education}\label{edukacnehry:hryine:mc}
Minecraft education je edukačná verzia klasickej hry Minecraft. 
V tejto hre sa dá naučiť viacero premetov, od fyziky, chémie až po geografiu.
Minecraft education je jedna z hier, ktorá je aktívne využívaná v základných školách vo Švédsku.
\subsubsection{World of Zoo}
Už z názvu je jasné, že v tejto hre sa hráč naučí niečo viac z biológie. 
V tejto hre je zakomponovaných viac než 40 druhov zvierat. V hre sú uvedené informácie o týchto zvieratách - kde žijú, ako sa stravujú, aký majú spôsob života, atd..
\subsubsection{Democracy} 
V tejto hre sa hráč naučí niečo o politických systémoch. Ako fungujú, aké sú ich princípy. 
Hráč má možnosť zobrať rolu prezidenta, či premiéra a vidieť, ako jeho rozhodnutia budú ovplyvňovať chod krajiny.





\section{Záver} \label{zaver} % prípadne iný variant názvu
Vďaka hrám je možné lepšie monitorovať a kontrolovať proces výučby. Počítačové hry dokážu zlepšiť kognitívne schopnosti a taktiež rozvinúť matematické a počítačové zručnosti. Pomocou počítačových hier sa dajú naučiť nielen fakty, ale aj spôsoby riešenia zložitých problémov. Taktiež rozvíjajú kreativitu a dokážu poskytnúť priestor pre vysvetlenie princípov, ktoré sa inak ťažko vysvetľujú. Do tejto skupiny patria napríklad fakty a práca s nebezpečnými chemickými látkami. Samozrejme okrem výhod majú hry aj nevýhody. Momentálne nemáme dostatok vhodných počítačových hier pre výučbu. Počítačové hry sú spojené s dlhým sedením pri počítači, čo je spojené s viacerými zdravotnými problémami. 
Celkovo má ale vzdelávanie pomocou počítačových hier viac výhod ako nevýhod. Počítačové hry dokážu naučíť rovnaké fakty novým, lepším spôsobom.



%\acknowledgement{Ak niekomu chcete poďakovať\ldots}


% týmto sa generuje zoznam literatúry z obsahu súboru literatura.bib podľa toho, na čo sa v článku odkazujete
\bibliography{literatura}
\bibliographystyle{plain} % prípadne alpha, abbrv alebo hociktorý iný
\end{document}
